
\chapter{Frequently asked Questions}

\section*{How easy is it to hack into my programs?}

Well, first of all: it is possible. The basic source files (.bas) are of course
readable by any text editor and such modifiable. The bytecode compiled code (.b)
is already harder to read and nearly impossible to convert back into source
code. However, since X11-basic is open source, everybody who wants to can look
into the sourcecode and can read all information necessary to decode the
bytecode and also modify it. It's possible but a real big job to do. On the level
of bytecode translated to C source also here someone could modify it. Once the
bytecode is compiled into real machine language, the code is as safe from
hackers as any other code is (means that there is nearly no way back). 

Even if you incorporate the bytecode into the virtual machine, your program
should be safe from snoopers, they might not even know your program is byte-code
generated. You can also instruct the bytecode compiler not to attach any symbol
table or extra debugging information.


\section*{Do I need a license to distribute my programs?}

No. You don't need a license to use X11-Basic (it's free), and you definitely
don't need any license to distribute or sell your programs. The only agreement
you have to worry about is that if you choose to use X11-Basic, you assume any
and all consequences, direct or indirectly from the use of X11-Basic. Which
means: don't blame me if it doesn't work as you think it should. X11-Basic can
be used for any task, whether it's profit-seeking or otherwise. I do not want to
know, and you don't pay me a cent. You don't even have to acknowledge that your
program was created with X11-Basic (although this would be a nice gesture).
You're allowed to bundle X11-Basic along with your program(s), as long as the
user is well informed that it's not buying into X11-Basic, but rather, buying
into your program. How is that done? By not even advertising that your
distribution includes a copy of X11-Basic. However, if you want to distribute or
modify X11-Basic itself, or if you want to incorporate parts of the X11-Basic 
sourcecode, you will need to follow the GNU public license. 

%Speed of X11-Basic
\section*{How fast is X11-Basic?}

The answer depends on the way an X11-Basic program is
run: It depends on if the code is interpreted, run as bytecode in a virtual
machine, or being compiled to native machine language. Generally we find:

\begin{enumerate}
\item X11-Basic programs run by the interpreter are slow,
\item X11-Basic programs compiled to bytecode and then run in the X11-Basic 
virtual machine (\verb|xbvm|) is fast, but
\item X11-Basic bytecode compiled natively to real machine language 
is even faster.
\item arbitrary precision numbers and calculations are slow, but 
\item 64bit floating point and complex number calculations as well as 
32bit integers are very fast.
\end{enumerate}

Bytecoded programs are always interpreted faster than scripted programming 
languages. The X11-Basic compiler can translate the X11-Basic bytecode to C, 
which then can be compiled to native machine language using any C-compiler 
(preferably \verb|gcc| on UNIX systems). Obviously your programs will be slower
than optimized C/C++ code but it already comes close.

If you need highest possible speed you can load and link a separate
DLL/shared object with the time critical part of your code written in another 
language (e.g. C or Assembler). 

A speed comparison was done with the Whetstone benchmark ($\longrightarrow$
\verb|Whets.bas|). This shows, that bytecode-programs are about 19 times faster
than the interpreted code and a natively compiled program can run about 28 times
faster.



\section*{UTF-8 character set}

I downloaded the last update to X11-Basic but I have a problem 
with the UTF-8 character set... I cannot use no more the 
ascii set, especially the graphic part of it ...
I made a small game that use them now it is not working no more...
Is there a way to fix this problem?

  A: Yes, there is. All characters are still there, but you cannot 
     access them with a simple \verb|CHR$()|. One method is to 
     copy the characters from a unicode table like this: 
{\footnotesize\begin{verbatim}
  [http://de.wikipedia.org/wiki/Unicodeblock\_Rahmenzeichnung Frames] 
\end{verbatim} }
     with the mouse into the editor. 
     You need to use a UTF-8 capable editor, e.g. pico, nano, gedit.
     If this is not working for you, alternatively you can code the 
     character yourself by the unicode number:

  {\footnotesize\begin{verbatim}
     FUNCTION utf8$(unicode%)
       IF unicode%<0x80
         RETURN CHR$(unicode%)
       ELSE IF unicode%<0x800
         RETURN CHR$(0xc0+(unicode%/64 AND 0x1f))+CHR$(0x80+(unicode% AND 0x3f))
       ELSE
         RETURN CHR$(0xe0+(unicode%/64/64 AND 0xf))+CHR$(0x80+(unicode%/64 AND 0x3f))+ \
                CHR$(0x80+(unicode% AND 0x3f))  
       ENDIF
     ENDFUNCTION
 \end{verbatim} }

     So e.g. the charackter 0x250C can be coded with \verb|@utf8$(0x250C)|.


\section*{GUI-Designer}
Is there a GUI-Designer for the graphical user unterface functions of X11-Basic?

 A: Well, so far nobody has made a real efford to write a real graphical  GUI designer.
    But the program \verb|gui2bas| may help creating GUI forms.  The input is a very siple
    ASCII-File (*.gui) which defines the interface.  So far many GEM object types are
    supportet (and even Atart ST \verb|*.rsc|-files  may be converted to \verb|*.gui|
    files with the \verb|rsc2gui| program.) but support for listboxes, popup-menues and
    Tooltips may be included in future.

\section*{Others}
\begin{verbatim}
 Q: My old ANSI Basic Programs (with line-Numbers) produce lots 
    of errors in the interpreter. How can I run classic 
    (ANSI) Basic programs?
 A: Classic Basic programs have to be converted before they can 
    be run with X11-Basic. With the bas2x11basic converter 
    program most of this convertion will be done automatically. 
\end{verbatim}


